% Template LaTeX file for MInfSem papers
%
% To generate the correct references using BibTeX, run
%     latex, bibtex, latex, latex
% modified...
% - from DAFx-00 to DAFx-02 by Florian Keiler, 2002-07-08
% - from DAFx-02 to DAFx-03 by Gianpaolo Evangelista
% - from DAFx-05 to DAFx-06 by Vincent Verfaille, 2006-02-05
% - from DAFx-06 to DAFx-07 by Vincent Verfaille, 2007-01-05
%                          and Sylvain Marchand, 2007-01-31
% - from DAFx-07 to DAFx-08 by Henri Penttinen, 2007-12-12
%                          and Jyri Pakarinen 2008-01-28
% - from DAFx-08 to DAFx-09 by Giorgio Prandi, Fabio Antonacci 2008-10-03
% - from DAFx-09 to DAFx-10 by Hannes Pomberger 2010-02-01
% - from DAFx-10 to DAFx-12 by Jez Wells 2011
% - from DAFx-12 to DAFx-14 by Sascha Disch 2013
% - from DAFx-14 to MInfSem by Wolfgang Fohl 2014
%
% Template with hyper-references (links) active after conversion to pdf
% (with the distiller) or if compiled with pdflatex.
%
% 20060205: added package 'hypcap' to correct hyperlinks to figures and tables
%                      use of \papertitle and \paperauthorA, etc for same title in PDF and Metadata
%
% 1) Please compile using latex or pdflatex.
% 2) If using pdflatex, you need your figures in a file format other than eps! e.g. png or jpg is working
% 3) Please use "paperftitle" and "pdfauthor" definitions below

%------------------------------------------------------------------------------------------
%  !  !  !  !  !  !  !  !  !  !  !  ! user defined variables  !  !  !  !  !  !  !  !  !  !  !  !  !  !
% Please use these commands to define title and author of the paper:
\def\papertitle{Herausforderungen bei open source Intelligence (OSINT)}
\def\paperauthorA{Lennart Karsten}


%------------------------------------------------------------------------------------------
\documentclass[twoside,a4paper]{article}
\usepackage{minfsem}
\usepackage{amsmath,amssymb,amsfonts,amsthm}
\usepackage{euscript}
\usepackage[utf8]{inputenc}
\usepackage[T1]{fontenc}
\usepackage{ifpdf}
\usepackage[ngerman]{babel}
\usepackage[babel,german=quotes]{csquotes} % Für gute Anführungszeichen
\usepackage{caption}
\usepackage{subfig, color}

\setcounter{page}{1}
\ninept

\usepackage{times}
% Saves a lot of ouptut space in PDF... after conversion with the distiller
% Delete if you cannot get PS fonts working on your system.

% pdf-tex settings: detect automatically if run by latex or pdflatex
\newif\ifpdf
\ifx\pdfoutput\relax
\else
   \ifcase\pdfoutput
      \pdffalse
   \else
      \pdftrue
\fi

\ifpdf % compiling with pdflatex
  \usepackage[pdftex,
    pdftitle={\papertitle},
    pdfauthor={\paperauthorA},
    colorlinks=false, % links are activated as colror boxes instead of color text
    bookmarksnumbered, % use section numbers with bookmarks
    pdfstartview=XYZ % start with zoom=100% instead of full screen; especially useful if working with a big screen :-)
  ]{hyperref}
  \pdfcompresslevel=9
  \usepackage[pdftex]{graphicx}
  \usepackage[figure,table]{hypcap}
\else % compiling with latex
  \usepackage[dvips]{epsfig,graphicx}
  \usepackage[dvips,
    colorlinks=false, % no color links
    bookmarksnumbered, % use section numbers with bookmarks
    pdfstartview=XYZ % start with zoom=100% instead of full screen
  ]{hyperref}
  % hyperrefs are active in the pdf file after conversion
  \usepackage[figure,table]{hypcap}
\fi

\title{\papertitle}

%--------------AUTHOR HEADER STARTS -----------------------
\affiliation{
\paperauthorA,}% \sthanks{This work was supported by the XYZ Foundation}}
{\href{http://www.haw-hamburg.de/ti-i}{Hamburg University of Applied Sciences,
    Dept. Computer Science,} \\ Berliner Tor 7\\ 20099 Hamburg, Germany\\
{\ttfamily \href{mailto:your.name@haw-hamburg.de}{lennart.karsten@haw-hamburg.de}}
}
%-----------------------------------AUTHOR HEADER ENDS------------------------------------------------------

\begin{document}
% more pdf-tex settings:
\ifpdf % used graphic file format for pdflatex
  \DeclareGraphicsExtensions{.png,.jpg,.pdf}
\else  % used graphic file format for latex
  \DeclareGraphicsExtensions{.eps}
\fi

\maketitle

\begin{abstract}
Durch den massiven Anstieg in der Nutzung von Social Media Plattformen, wie Twitter oder Facebook, hat dessen Wichtigkeit für die Gewinnung von \textit{Business intelligence} Informationen in den vergangenen Jahren stark zugenommen. \\
Die Erhebung von Daten aus solchen, meist frei zugänglichen Quellen ist deutlich kostengünstiger als die Erhebung aus klassischen Quellen. Dies hat dazu geführt, dass Open Source Intelligence (OSINT) mittlerweile die wichtigste Quelle zu Gewinnung von \textit{Business intelligence} ist.\\
Im Gegensatz zu klassischen Aggregierungsmethoden ist die Beschaffung einer ausreichenden Menge an Daten beim OSINT kein Problem. Probleme Ergeben sich durch große Mengen an Daten, die teilweise widersprüchlich oder falsch sein können. \\
Die Ausarbeitung befasst sich mit der Beschreibung solcher Problemstellungen und nennt verschiedene Lösungsansätze.
\end{abstract}

\section{Einleitung}
Das Finden von Informationen zur Entscheidungsfindung war noch bis vor ein paar Jahren eine aufwendige Aufgabe. Um ein umfassendes Bild über eine Personengruppe oder eine einzelne Person zu erlangen mussten eine Vielzahl von Quellen ausgewertet werden. Der Vorgang war zeitintensiv und somit teuer.\\
Durch die alltäglichen Nutzung des Internets durch Millionen von Personen hat sich das Bild gewandelt. Es ist mittlerweile möglich Daten massenhaft zu speichern und auszuwerten. Die Qualität der Daten hat mit dieser neuen Methode jedoch massiv abgenommen. Dies ergibt sich daraus, dass jeder Benutzer beliebige Daten im Netz über sich streuen kann. Um Die Daten auswerten zu können müssen diese vergleichbar gemacht werden. Dennoch muss bedacht werden, dass Ergebnisse aus einer derartigen elektronischen Auswertung niemals zu 100\% korrekt sein können.

\section{Vergleichbare Arbeiten}
\cite{challanges_in_osint}
\enquote{Challenges in Open Source Intelligence}\\
Beschreibt allgemeine Schwierigkeiten bei OSINT.\par

\noindent\cite{challenges_to_automated_alloegory}
\enquote{Challenges to Automated Allegory Resolution in Open Source Intelligence}\\
Beschreibt Probleme bei der Auswertung von Sprache\par

\noindent\cite{development_of_a_hybrid_decision_system}
\enquote{Development of a Hybrid Decision Support System for intelligence analysis}\\
Beschreibt ... \par 

\noindent\cite{data_consolidation_solution}
\enquote{Data consolidation solution for internal security needs}\\
Beschreibt ... .


\section{Hauptteil}



\section{Zusammenfassung}


%\newpage
\nocite{*}
\bibliographystyle{IEEEbib}
\bibliography{MI-Hausarbeit} % requires file minfsem_tmpl.bib

%\section{Appendix: Margin Check}
%This section shows the column margins for the text. \bigskip\newline

\end{document}
